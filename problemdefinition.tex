\section{Problem Definition and Previous Research}
We aim to detect software vulnerability with sufficient accuracy and granularity.
In this section we describe our problem domain in detail.
We also introduce previous research such as static analysis or bug detection,
and claim the need for a new means of vulnerability detection by showing their limitations.

% High level에서 우리가 무슨 문제를 해결하려 하는지
% 구체적인 내용을 설명을 하겠다- / 왜 기존 방법이 적절하지 않은지 말하겠다-

\subsection{Problem Definition}

Many large softwares contain more or less vulnerabilities. In linux kernel, hundreds of vulnerabilities are found each year \cite{cvelinux}.
How to detect these vulnerability in software has been studied long and diverse methods has been introduced.
Previous methods include static or dynamic analysis, formal verfication, or 

\subsubsection{Static Analysis}

KLEE

\subsubsection{Dynamic Analysis}

\subsubsection{Formal Verification}

Coq

\subsubsection{Bug Detection}

Software defect prediction is one of classic bug detection methods.
Approaches using machine learning have been applied to improve the accuracy of prediction.
Feature subset selection. [Gao et al.]
Feature generation. [Song et al.]

% 취약점은 버그의 일종인데 이런게 있는데 완전 별개의 문제이다
% Symbolic execution 같은 걸 넣고 bug detection 비중은 낮추는 게

However software metrics used for prediction are rater naive to detect vulnerability
with high accuracy and granularity.

In order to reflect more sophisticated attributes of software, other means of representation and algorithm is required.
