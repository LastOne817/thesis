\section{Problem Definition and Previous Research}
We aim to detect software vulnerability with sufficient accuracy and granularity.
In this section we describe our problem domain in detail.
We also introduce typical previous research such as static analysis or bug detection,
and motivate the need for a new means of vulnerability detection by showing their limitations.

% High level에서 우리가 무슨 문제를 해결하려 하는지
% 구체적인 내용을 설명을 하겠다- / 왜 기존 방법이 적절하지 않은지 말하겠다-

\subsection{Problem Definition}

Many large softwares contain more or less vulnerabilities. In linux kernel, hundreds of vulnerabilities are found each year \cite{cvelinux}.
How to detect these vulnerability in software has been studied long and diverse methods has been introduced.
Previous methods include static or dynamic analysis, formal verfication, and a variant of bug detection methods.

\subsection{Static Analysis}

Static analysis is a way of analysing software without running the software itself.
This includes symbolic execution such as KLEE \cite{cadar2008klee}, which determines what input leads to certain part of the program by using symbolic values for variables.
It can detect vulnerability with specific attack vector and clearly show how it exploits the program.
However symbolic execution has limitation known as path explosion which makes using symbolic execution on large softwares practically infeasible.

Other methods of static analysis includes formal verification, which proves program specification based on mathematical design.
Although being applied in recent softwares \cite{rustbelt}, formal verification has difficulty of designing program specifications into mathematical form and proving them, both of which cannot be done in automated way.

\subsection{Dynamic Analysis}

In contrast to static analysis, dynamic analysis executes the software for analysis and observes its behaviour.
Dynamic analysis can be used in memory error detection \cite{valgrind}, software testing associated with code coverage \cite{huang2015code},
or analysing program behaviour \cite{newsome2005dynamic, enck2014taintdroid}.
However dynamic analysis has limited scope of detection and cannot consider every possible attack vectors.

\subsection{Bug Detection}

Bug detection is a different problem domain from vulnerability detection, but due to their relevance some bug detection approaches have been applied to finding vulnerabilities.
Software defect prediction \cite{khoshgoftaar2009attribute, lessmann2008benchmarking} is one of classic bug detection methods.
As a metric-based method it uses code features such as lines of code(LoC) or function call dependency.

Approaches using machine learning are currently being applied to improve the accuracy of prediction \cite{gao2011choosing, wang2016automatically}.
They can effectively detect bugs even in large softwares, but since they rely on overall metrics the granularity of detection is mostly limited to file or module level.
In order to reflect more sophisticated attributes of software, other means of representation and algorithm is required.

% 취약점은 버그의 일종인데 이런게 있는데 완전 별개의 문제이다
% Symbolic execution 같은 걸 넣고 bug detection 비중은 낮추는 게