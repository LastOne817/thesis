\section{Problem Definition and Previous Research}
We aim to detect software vulnerability with sufficient accuracy and granularity.
In this section we describe our problem domain in detail.
We also introduce previous research such as static analysis or bug detection,
and claim the need for a new means of vulnerability detection by showing their limitations.

% High level에서 우리가 무슨 문제를 해결하려 하는지
% 구체적인 내용을 설명을 하겠다- / 왜 기존 방법이 적절하지 않은지 말하겠다-

\subsection{Problem Definition}

How to detect vulnerability in software has been studied long and diverse methods has been introduced.

\subsection{Static Analysis}

KLEE

\subsection{Program Verification}

Coq

\subsection{Bug Detection}

Software defect prediction is one of classic bug detection methods.
Approaches using machine learning have been applied to improve the accuracy of prediction.
Feature subset selection. [Gao et al.]
Feature generation. [Song et al.]

% 취약점은 버그의 일종인데 이런게 있는데 완전 별개의 문제이다
% Symbolic execution 같은 걸 넣고 bug detection 비중은 낮추는 게

However software metrics used for prediction are rater naive to detect vulnerability
with high accuracy and granularity.

In order to reflect more sophisticated attributes of software, other means of representation and algorithm is required.

\eat{
Vulnerability detection has been developed in the form of software defect prediction.
Software defect prediction is a metric-based method to detect bugs in softwares. 
It is based on the ground assumption: the more complex the code, the more it is defect-prone.
Defect prediction represents software with metrics of product and process,
such as number of procedure calls, number of lines of code, or number of changes to the code.

Classic defect prediction methods have been developed in terms of introducing new metrics.
Current defect prediction methods make use of machine learning techniques.
Feature subset selection aims to find optimal set of metrics for defect prediction.
Feature generation aims to generate new metrics for defect prediction with Deep Belief Network.
 
However, software metrics used for defect prediction are rather naive to
detect vulnerability with high accuracy and granularity.
For instance, vulnerabilities like buffer overflow cannot be detected accurately with simple code metrics
since they can occur regardless of code complexity.
Metric-based method also cannot determine the type of detected vulnerability.
It also lacks granularity, as most previous researches provide module-level prediction.

In order to reflect more sophisticated attributes of software, other means of representation and algorithm is required.
}