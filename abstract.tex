\subsection*{Abstract}

\eat{
As the number of softwares being produced and the size of each softwares grows, vulnerabilities in software are also becoming more prevalent.
Detecting such vulnerabilities in software has been worked on for a long time and various methods have been introduced, for instance static or dynamic program analysis.
Although they had achived their own success, accuracy and granularity needs to be improved to efficienty detect vulnerabilities.
% 예시 등 연결사
Given the growing complexity and quantity of modern software, detection technique should also be automated.
% 기존 방법도 automation 되어있다고 봐야할거같은데
% 2번째 문장으로 고쳐서 연결시키고
% Detecting ~을 그에 맞게 적당히 수정
To achieve such objectives, we propose new vulnerability detection method combining comprehensive representation and deep learning.
% new는 빼야할듯
% 아니 그냥 두자
We suggest code property graph as representation, and compare various learning models to reach our objectives.
% propose - suggest 로 연결하지 말고(중복), 활용할 거라고 쓰던가 하는 방식으로
}

As the number of softwares being produced and the size of each software grow, vulnerabilities in softwares are also becoming more prevalent.
In order to detect these vulnerabilities, efficient detection techniques are required.
How to detect vulnerabilities has been worked on for a long time and various approaches have been introduced, for instance, static or dynamic program analysis.
Although they had achieved their own success, they are still practically infeasible to be applied to complex modern softwares.
Various requirements such as modest execution time, accuracy and granularity should be met to efficiently detect vulnerabilities.
To achieve such objectives, I propose vulnerability detection method combining comprehensive representation and deep learning.
I use code property graph as such representation, and compare various deep learning models to reach our objectives.
