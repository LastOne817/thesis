\section{Approach}

To address the limitations identified in the previous section we need better representation which contains comprehensive semantics, other than just code metrics.
% we need better ~ 대신 'DL을 쓸거고 새로운 representation을 쓸거다'
% 왜 DL을 쓰는지, 왜 이런 representation을 써야 하는지 

\begin{itemize}
\item
First, to achieve function-level granularity, basic structure of source code should be included.
% LoC 같은걸론 파악할 수 없다는 걸 말하기 위함인데 부가설명이 필요한듯
% dawn song 논문에 DL을 써야하는 이유
% => DL을 쓰는데 representation에 이러이런게 필요하다 => CPG

\item
Control flow and dependence of program are also required, since same code can be either safe or vulnerable depending on the context.
\end{itemize}

We use code property graph which combines abstract syntax tree, control flow graph, and program dependency graph,
therefore comprehensively represents our requirements mentioned above.
% CPG 자체에 대한 언급은 아래 subsection으로. 조건만

\subsection{Representation}

Code property graph is introduced to effectively mine large amounts of source code for vulnerabilities.

\begin{itemize}
\item
AST
\item
CFG
\item
PDG
\end{itemize}
\subsection{Neural Network for Graph Data}

Since we will use CPG as program vector representation, we select learning models among which takes graph data as input.

\begin{itemize}
\item
Graph kernel

\item
Graph Neural Network

\item
Convolutional Neural Network

\begin{itemize}

\item
Convolutional neural network is typical deep learning model used widely in image recognition. Typically convolutional neural network takes 2-dimensional data, usually image, as input and feed forward them in network with convolution operations.
\end{itemize}

\end{itemize}